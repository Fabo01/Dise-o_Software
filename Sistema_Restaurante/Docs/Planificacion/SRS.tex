\documentclass[12pt]{article}
\usepackage[utf8]{inputenc}
\usepackage[spanish]{babel}
\usepackage{geometry}
\geometry{a4paper, margin=2.5cm}
\usepackage{graphicx}
\usepackage{hyperref}
\usepackage{enumitem}
\usepackage{longtable}
\usepackage{titlesec}
\usepackage{fancyhdr}
\pagestyle{fancy}

\lhead{Sistema de Gestión de Restaurantes}
\rhead{Especificación de Requisitos}

\title{Especificación de Requisitos de Software (SRS)}
\author{Equipo de Desarrollo}
\date{Octubre 2023}

\begin{document}

\maketitle

\tableofcontents

\newpage

\section{Introducción}
\subsection{Propósito}
Este documento tiene como objetivo definir y documentar de forma clara y detallada los requisitos del Sistema de Gestión de Clientes y Pedidos en un Restaurante. Servirá como base para la planificación, diseño, desarrollo, validación y mantenimiento del sistema, dirigido a desarrolladores, evaluadores y stakeholders del proyecto.

\subsection{Alcance}
El sistema es una aplicación web destinada a modernizar la gestión operativa de restaurantes, migrando de una solución actual de escritorio (Python + customtkinter) a una plataforma web moderna (React + Django). El sistema permitirá:
\begin{itemize}
  \item Gestionar clientes y sus pedidos.
  \item Administrar inventario de ingredientes.
  \item Crear y actualizar menús vinculados a los ingredientes disponibles.
  \item Gestionar mesas y tiempos de ocupación.
  \item Generar reportes analíticos para la toma de decisiones.
  \item Implementar un sistema de delivery para expandir canales de venta.
  \item Administrar diferentes medios de pago y comprobantes.
\end{itemize}

La transformación no solo modernizará la infraestructura tecnológica, sino que también transformará fundamentalmente el modelo operativo del restaurante, mejorando la eficiencia, reduciendo costos y aumentando la satisfacción del cliente.

\subsection{Definiciones, Acrónimos y Abreviaturas}
\begin{itemize}
  \item \textbf{SRS}: Software Requirements Specification (Especificación de Requisitos de Software).
  \item \textbf{RF}: Requisito Funcional.
  \item \textbf{RNF}: Requisito No Funcional.
  \item \textbf{GUI}: Graphical User Interface (Interfaz Gráfica de Usuario).
  \item \textbf{API}: Application Programming Interface (Interfaz de Programación de Aplicaciones).
  \item \textbf{CRUD}: Create, Read, Update, Delete (Crear, Leer, Actualizar, Eliminar) - Operaciones básicas sobre datos.
  \item \textbf{ORM}: Object-Relational Mapping (Mapeo Objeto-Relacional).
  \item \textbf{JWT}: JSON Web Token - Estándar para la creación de tokens de acceso.
  \item \textbf{KPI}: Key Performance Indicator (Indicador Clave de Desempeño).
  \item \textbf{Stock}: Cantidad disponible de un ingrediente en el inventario.
  \item \textbf{Delivery}: Servicio de entrega de pedidos a domicilio.
  \item \textbf{Frontend}: Parte del sistema con la que interactúan directamente los usuarios.
  \item \textbf{Backend}: Parte del sistema que procesa las operaciones y gestiona los datos.
\end{itemize}

\subsection{Referencias}
\begin{itemize}
  \item IEEE Std 830-1998, \emph{Recommended Practice for Software Requirements Specifications}.
  \item Documentación del sistema actual de escritorio en Python.
  \item Diagramas de Casos de Uso del sistema actual y refactorizado.
  \item Casos de Uso Extendidos - Documentación detallada de flujos de trabajo.
  \item Sistematización del Proyecto - Análisis de la situación sin y con proyecto.
  \item Convenciones de Código y Estructura del Proyecto.
  \item Backlog de tareas para el desarrollo del proyecto.
\end{itemize}

\subsection{Visión General del Documento}
Este documento se estructura en las siguientes secciones:
\begin{itemize}
  \item \textbf{Introducción}: Contexto, propósito y alcance del documento.
  \item \textbf{Descripción General}: Visión de alto nivel del sistema y sus características.
  \item \textbf{Requisitos Específicos}: Detalle de los requisitos funcionales y no funcionales.
  \item \textbf{Modelo de Datos}: Estructura de almacenamiento y relaciones entre entidades.
  \item \textbf{Arquitectura del Sistema}: Componentes principales y relaciones entre ellos.
  \item \textbf{Límites del Sistema}: Alcance y restricciones funcionales y técnicas.
  \item \textbf{Planificación del Proyecto}: Metodología, cronograma y organización del trabajo.
  \item \textbf{Resultados Esperados}: Beneficios y entregables concretos del proyecto.
  \item \textbf{Apéndices}: Información complementaria, glosario y diagramas.
\end{itemize}

\section{Descripción General}
\subsection{Perspectiva del Producto}
El sistema es una aplicación cliente-servidor distribuida que se integrará a la infraestructura existente del restaurante. Permitirá acceso desde múltiples dispositivos y ubicaciones, optimizando la operación diaria y mejorando la experiencia del cliente.

El sistema refactorizado sigue un enfoque de arquitectura limpia (Clean Architecture) que separa claramente las capas de:
\begin{itemize}
  \item \textbf{Presentación}: Interfaces de usuario en React y controladores de API.
  \item \textbf{Casos de Uso}: Lógica de negocio específica de la aplicación.
  \item \textbf{Dominio}: Entidades y reglas de negocio core.
  \item \textbf{Infraestructura}: Acceso a datos, frameworks y servicios externos.
\end{itemize}

Esta arquitectura mejora significativamente respecto al sistema actual, que tiene una estructura monolítica con acoplamiento fuerte entre interfaz y lógica de negocio.

\subsection{Funcionalidad del Producto}
El sistema permitirá:

\subsubsection{Gestión de Clientes}
\begin{itemize}
  \item Registro y actualización de datos personales.
  \item Búsqueda avanzada por diferentes criterios.
  \item Historial de pedidos y preferencias.
  \item Gestión descentralizada que permite a meseros crear y actualizar datos.
\end{itemize}

\subsubsection{Gestión de Inventario}
\begin{itemize}
  \item Control detallado de ingredientes con alertas de stock.
  \item Actualización en tiempo real al procesar pedidos.
  \item Categorización y trazabilidad de ingredientes.
  \item Informes de uso y tendencias de consumo.
\end{itemize}

\subsubsection{Gestión de Menús}
\begin{itemize}
  \item Creación y actualización de platos con sus ingredientes.
  \item Cálculo automático de costos y márgenes de beneficio.
  \item Verificación automática de disponibilidad según stock.
  \item Categorización y personalización de platos.
\end{itemize}

\subsubsection{Gestión de Pedidos}
\begin{itemize}
  \item Creación, modificación y cancelación de pedidos.
  \item Seguimiento del estado en tiempo real.
  \item Comunicación directa entre servicio y cocina.
  \item Notificaciones automáticas sobre cambios de estado.
\end{itemize}

\subsubsection{Gestión de Mesas}
\begin{itemize}
  \item Visualización gráfica del estado de las mesas.
  \item Control de tiempos de ocupación y rotación.
  \item Asignación óptima según tamaño de grupos.
  \item Estadísticas de uso y eficiencia.
\end{itemize}

\subsubsection{Sistema de Delivery}
\begin{itemize}
  \item Registro y seguimiento de pedidos a domicilio.
  \item Integración con plataformas externas de delivery.
  \item Asignación y seguimiento de repartidores.
  \item Gestión de zonas de cobertura y tiempos de entrega.
\end{itemize}

\subsubsection{Sistema de Pagos}
\begin{itemize}
  \item Procesamiento de múltiples medios de pago.
  \item Generación de comprobantes fiscales.
  \item División de cuentas y propinas.
  \item Registro histórico de transacciones.
\end{itemize}

\subsubsection{Reportes Analíticos}
\begin{itemize}
  \item Dashboard interactivo con KPIs del negocio.
  \item Visualización de tendencias y patrones.
  \item Exportación de informes en múltiples formatos.
  \item Análisis predictivo de demanda y stock necesario.
\end{itemize}

\subsection{Características de los Usuarios}
El sistema contempla los siguientes roles de usuario:

\begin{itemize}
  \item \textbf{Jefe de Local (Administrador):} Acceso completo a todas las funcionalidades del sistema. Responsable de la configuración, parametrización y supervisión general. Tiene acceso a 33 casos de uso (100\% del sistema).
  
  \item \textbf{Jefe de Turno:} Responsable de la operación diaria y análisis de datos. Acceso a funcionalidades administrativas pero con ciertas restricciones en configuración. Tiene acceso a 17 casos de uso (52\% del sistema).
  
  \item \textbf{Mesero:} Personal encargado de atender a los clientes, tomar pedidos y gestionar mesas. Acceso a funcionalidades de servicio y atención al cliente. Tiene acceso a 16 casos de uso (48\% del sistema).
  
  \item \textbf{Cocina:} Personal encargado de preparar los pedidos y gestionar ingredientes. Acceso limitado a visualización de pedidos e inventario. Tiene acceso a 6 casos de uso (18\% del sistema).
\end{itemize}

Esta distribución de roles y permisos representa una mejora significativa respecto al sistema actual, donde todas las operaciones están centralizadas en el Jefe de Local.

\subsection{Restricciones}
\begin{itemize}
  \item \textbf{Tecnológicas:} El sistema debe utilizar React para el frontend y Django para el backend.
  \item \textbf{Seguridad:} Implementación de autenticación JWT, encriptación de datos sensibles y protección contra ataques comunes.
  \item \textbf{Rendimiento:} El sistema debe procesar operaciones estándar en menos de 3 segundos.
  \item \textbf{Conectividad:} El sistema debe funcionar en entornos con conectividad intermitente, manejando operaciones offline cuando sea necesario.
  \item \textbf{Legales:} Cumplimiento con normativas de protección de datos personales y requisitos fiscales para la emisión de comprobantes.
  \item \textbf{Hardware:} Compatibilidad con dispositivos móviles y tablets para el personal de servicio.
\end{itemize}

\subsection{Suposiciones y Dependencias}
\begin{itemize}
  \item \textbf{Suposiciones:}
  \begin{itemize}
    \item Los usuarios tendrán capacitación básica en tecnología.
    \item El restaurante cuenta con una red Wi-Fi estable.
    \item El volumen de datos no excederá inicialmente los 10GB.
    \item Los picos de carga no superarán 100 usuarios concurrentes en la primera fase.
  \end{itemize}
  
  \item \textbf{Dependencias:}
  \begin{itemize}
    \item Disponibilidad continua del servidor de base de datos PostgreSQL.
    \item Integración con servicios de pago electrónico de terceros.
    \item APIs de plataformas de delivery para la integración de pedidos externos.
    \item Servicio de mapas para visualización de rutas de entrega.
  \end{itemize}
\end{itemize}

\subsection{Requisitos Futuros}
\begin{itemize}
  \item Aplicación móvil dedicada para clientes finales.
  \item Sistema de fidelización con puntos y recompensas.
  \item Integración con sistemas de reserva online.
  \item Módulo de gestión de proveedores y compras.
  \item Análisis predictivo de demanda basado en históricos y factores externos.
  \item Integración con sistemas de marketing digital.
\end{itemize}

\section{Requisitos Específicos}
\subsection{Requisitos Funcionales (RF)}

\subsubsection{RF-1: Gestión de Clientes}
\begin{itemize}
  \item \textbf{Descripción:} Permitir el registro, edición y eliminación de clientes.
  \item \textbf{Entrada:} Datos del cliente (nombre, email, teléfono, dirección).
  \item \textbf{Proceso:} Validación de datos, verificación de duplicados, almacenamiento en base de datos.
  \item \textbf{Salida:} Confirmación de la acción o mensaje de error.
  \item \textbf{Actores:} Jefe de Local, Jefe de Turno, Mesero.
\end{itemize}

\subsubsection{RF-2: Gestión de Inventario}
\begin{itemize}
  \item \textbf{Descripción:} Registrar y actualizar el stock de ingredientes.
  \item \textbf{Entrada:} Datos del ingrediente (nombre, tipo, cantidad, unidad).
  \item \textbf{Proceso:} Validación de datos, actualización de inventario, generación de alertas.
  \item \textbf{Salida:} Confirmación de la acción o alerta de stock crítico.
  \item \textbf{Actores:} Jefe de Local, Jefe de Turno, Cocina.
\end{itemize}

\subsubsection{RF-3: Gestión de Pedidos}
\begin{itemize}
  \item \textbf{Descripción:} Crear y gestionar pedidos asociados a clientes y mesas.
  \item \textbf{Entrada:} Selección de menús, cliente y mesa.
  \item \textbf{Proceso:} Verificación de disponibilidad, cálculo de totales, actualización de inventario.
  \item \textbf{Salida:} Confirmación del pedido y generación de boleta.
  \item \textbf{Actores:} Jefe de Local, Jefe de Turno, Mesero.
\end{itemize}

\subsubsection{RF-4: Gestión de Mesas}
\begin{itemize}
  \item \textbf{Descripción:} Visualizar y actualizar el estado de las mesas.
  \item \textbf{Entrada:} Número de mesa, capacidad, estado.
  \item \textbf{Proceso:} Actualización de estado, cálculo de tiempos de ocupación.
  \item \textbf{Salida:} Confirmación de la acción o mensaje de error.
  \item \textbf{Actores:} Jefe de Local, Jefe de Turno, Mesero.
\end{itemize}

\subsubsection{RF-5: Reportes Analíticos}
\begin{itemize}
  \item \textbf{Descripción:} Generar reportes de ventas, uso de ingredientes y popularidad de menús.
  \item \textbf{Entrada:} Parámetros de filtro (fecha, categoría, etc.).
  \item \textbf{Proceso:} Análisis de datos históricos, generación de visualizaciones.
  \item \textbf{Salida:} Reportes en formatos PDF y gráficos interactivos.
  \item \textbf{Actores:} Jefe de Local, Jefe de Turno.
\end{itemize}

\subsubsection{RF-6: Sistema de Delivery}
\begin{itemize}
  \item \textbf{Descripción:} Gestionar pedidos para entrega a domicilio.
  \item \textbf{Entrada:} Datos de pedido, dirección de entrega, método de pago.
  \item \textbf{Proceso:} Verificación de cobertura, asignación de repartidor, seguimiento.
  \item \textbf{Salida:} Confirmación del pedido y estimación de tiempo de entrega.
  \item \textbf{Actores:} Jefe de Local, Jefe de Turno, Mesero.
\end{itemize}

\subsubsection{RF-7: Procesamiento de Pagos}
\begin{itemize}
  \item \textbf{Descripción:} Registrar pagos por diferentes métodos y generar comprobantes.
  \item \textbf{Entrada:} Método de pago, monto, datos fiscales si corresponde.
  \item \textbf{Proceso:} Validación de pago, registro en sistema, generación de comprobante.
  \item \textbf{Salida:} Confirmación de pago y comprobante fiscal.
  \item \textbf{Actores:} Jefe de Local, Jefe de Turno, Mesero.
\end{itemize}

\subsection{Requisitos No Funcionales (RNF)}

\subsubsection{RNF-1: Rendimiento}
El sistema debe ser capaz de procesar un considerable volumen de datos o de complejidades altas en un tiempo no mayor a 3 segundos. Las operaciones críticas como la toma de pedidos y actualización de inventario deben completarse en menos de 1 segundo.

\subsubsection{RNF-2: Seguridad}
\begin{itemize}
  \item Autenticación mediante credenciales y tokens JWT.
  \item Encriptación de datos sensibles en la base de datos.
  \item Protección contra ataques comunes como SQL Injection y XSS.
  \item Registro detallado de auditoría para operaciones críticas.
  \item Control de acceso basado en roles claramente definidos.
\end{itemize}

\subsubsection{RNF-3: Escalabilidad}
\begin{itemize}
  \item El sistema debe soportar múltiples usuarios simultáneamente sin degradación del rendimiento.
  \item Debe ser capaz de escalar horizontalmente mediante la adición de servidores.
  \item La arquitectura debe permitir el crecimiento modular de funcionalidades.
  \item Las operaciones de base de datos deben optimizarse con índices adecuados para volúmenes crecientes.
\end{itemize}

\subsubsection{RNF-4: Mantenibilidad}
\begin{itemize}
  \item La arquitectura debe permitir actualizaciones sin afectar la disponibilidad del servicio.
  \item El código debe estar documentado y seguir estándares de calidad.
  \item La implementación debe seguir principios SOLID y patrones de diseño reconocidos.
  \item Se deben implementar pruebas automatizadas con cobertura significativa.
\end{itemize}

\subsubsection{RNF-5: Disponibilidad}
El sistema debe garantizar una disponibilidad en equipos modernos y antiguos, siendo capaz de ejecutarse en entornos con menores recursos. Debe mantener funcionalidad básica incluso en condiciones de conectividad intermitente.

\subsubsection{RNF-6: Compatibilidad}
El sistema debe ser accesible desde navegadores modernos (Chrome, Firefox, Edge, Safari) y dispositivos móviles. La interfaz debe adaptarse a diferentes resoluciones de pantalla.

\subsubsection{RNF-7: Usabilidad}
\begin{itemize}
  \item La interfaz debe ser intuitiva y accesible para usuarios con conocimientos básicos de tecnología.
  \item El sistema debe proporcionar retroalimentación clara sobre las acciones realizadas.
  \item Los flujos de trabajo comunes deben completarse en un máximo de 3 interacciones.
  \item La curva de aprendizaje para nuevos usuarios debe ser mínima.
\end{itemize}

\subsection{Requisitos de Interfaces}

\subsubsection{Interfaces de Usuario}
\begin{itemize}
  \item \textbf{GUI Web:} Interfaz responsiva desarrollada en React, optimizada para diferentes dispositivos.
  \item \textbf{Dashboard:} Tablero interactivo con visualizaciones en tiempo real para análisis de datos.
  \item \textbf{Vista de Mesas:} Representación gráfica del restaurante con estados actualizados.
  \item \textbf{Panel de Cocina:} Interfaz simplificada que muestra pedidos pendientes ordenados por prioridad.
\end{itemize}

\subsubsection{Interfaces de Hardware}
\begin{itemize}
  \item \textbf{Computadoras:} Para administración y análisis de datos.
  \item \textbf{Tablets:} Para meseros y gestión de pedidos en mesa.
  \item \textbf{Pantallas táctiles:} Para cocina y visualización de pedidos pendientes.
  \item \textbf{Impresoras térmicas:} Para generación de comprobantes y comandas.
  \item \textbf{Dispositivos móviles:} Para seguimiento de entregas y notificaciones.
\end{itemize}

\subsubsection{Interfaces de Software}
\begin{itemize}
  \item \textbf{API REST:} Para comunicación entre frontend y backend.
  \item \textbf{Webhooks:} Para integraciones con plataformas de delivery.
  \item \textbf{SDKs:} Para procesadores de pago y servicios de mapas.
  \item \textbf{WebSockets:} Para actualizaciones en tiempo real de pedidos y estados.
\end{itemize}

\subsubsection{Interfaces de Comunicación}
\begin{itemize}
  \item \textbf{HTTP/HTTPS:} Protocolo principal de comunicación con TLS para seguridad.
  \item \textbf{WebSockets:} Para comunicaciones bidireccionales en tiempo real.
  \item \textbf{SMTP:} Para envío de notificaciones por correo electrónico.
  \item \textbf{SMS Gateway:} Para notificaciones a clientes sobre estado de pedidos.
\end{itemize}

\subsection{Precondiciones, Postcondiciones y Flujos Alternativos}

\subsubsection{Precondiciones Generales}
\begin{itemize}
  \item Usuario autenticado con nivel de permisos adecuado.
  \item Conexión al servidor disponible.
  \item Datos maestros necesarios ya configurados en el sistema.
\end{itemize}

\subsubsection{Postcondiciones Generales}
\begin{itemize}
  \item La operación se registra en el log de auditoría.
  \item Los datos modificados se actualizan en tiempo real para todos los usuarios.
  \item Se generan notificaciones relevantes para los actores involucrados.
\end{itemize}

\subsubsection{Flujos Alternativos Comunes}
\begin{itemize}
  \item \textbf{Error de validación:} Se muestran mensajes específicos y se permite corregir.
  \item \textbf{Problema de conexión:} Los datos se almacenan localmente y se sincronizan cuando se restablece.
  \item \textbf{Conflicto de datos:} Se notifica a los usuarios afectados y se proponen opciones de resolución.
  \item \textbf{Operación cancelada:} No se realizan cambios y se vuelve al estado anterior.
\end{itemize}

\section{Modelo de Datos}

\subsection{Diagrama Entidad-Relación}
El modelo de datos del sistema está diseñado para soportar todas las entidades principales y sus relaciones. Las principales entidades son:

\begin{itemize}
  \item \textbf{Cliente:} Almacena información de los clientes del restaurante.
  \item \textbf{Ingrediente:} Registra los insumos utilizados en la preparación de menús.
  \item \textbf{Menu:} Define los platos ofrecidos, con sus precios y categorías.
  \item \textbf{MenuIngrediente:} Establece la relación entre menús e ingredientes, con cantidades.
  \item \textbf{Pedido:} Registra los pedidos realizados por clientes.
  \item \textbf{PedidoMenu:} Relaciona pedidos con los menús seleccionados y sus cantidades.
  \item \textbf{Mesa:} Almacena información de las mesas del restaurante y su estado.
  \item \textbf{Delivery:} Registra información específica de pedidos a domicilio.
  \item \textbf{Pago:} Almacena transacciones financieras asociadas a pedidos.
  \item \textbf{Usuario:} Gestiona credenciales y permisos de los usuarios del sistema.
\end{itemize}

Ver diagrama completo en el Apéndice.

\subsection{Descripción de Tablas y Atributos}

\subsubsection{Cliente}
\begin{itemize}
  \item id: Identificador único
  \item rut: RUT o documento de identidad
  \item nombre: Nombre completo
  \item telefono: Número de contacto
  \item email: Correo electrónico
  \item direccion: Dirección física
  \item fecha\_registro: Fecha de primer registro
  \item ultima\_visita: Fecha de última interacción
\end{itemize}

\subsubsection{Ingrediente}
\begin{itemize}
  \item id: Identificador único
  \item nombre: Nombre del ingrediente
  \item categoria: Clasificación del ingrediente
  \item stock: Cantidad disponible
  \item unidad\_medida: Unidad de medida (kg, litros, etc.)
  \item nivel\_critico: Cantidad mínima antes de alertar
  \item precio\_unitario: Costo por unidad
  \item proveedor: Proveedor habitual
\end{itemize}

\subsubsection{Menu}
\begin{itemize}
  \item id: Identificador único
  \item nombre: Nombre del plato
  \item descripcion: Descripción detallada
  \item precio: Precio de venta
  \item categoria: Clasificación del menú
  \item disponible: Estado de disponibilidad
  \item tiempo\_preparacion: Tiempo estimado en minutos
  \item imagen: Ruta a la imagen del plato
\end{itemize}

\subsubsection{Pedido}
\begin{itemize}
  \item id: Identificador único
  \item cliente\_id: Relación con cliente
  \item mesa\_id: Relación con mesa (puede ser nulo para delivery)
  \item estado: Estado del pedido (recibido, preparando, listo, entregado, etc.)
  \item fecha\_hora: Timestamp de creación
  \item total: Monto total del pedido
  \item notas: Observaciones especiales
  \item usuario\_id: Usuario que registró el pedido
\end{itemize}

\subsection{Relaciones entre Entidades}
\begin{itemize}
  \item Un \textbf{Cliente} puede tener múltiples \textbf{Pedidos}.
  \item Un \textbf{Menu} puede contener múltiples \textbf{Ingredientes} (a través de \textbf{MenuIngrediente}).
  \item Un \textbf{Ingrediente} puede ser parte de múltiples \textbf{Menus} (a través de \textbf{MenuIngrediente}).
  \item Un \textbf{Pedido} puede incluir múltiples \textbf{Menus} (a través de \textbf{PedidoMenu}).
  \item Un \textbf{Pedido} puede estar asociado a una \textbf{Mesa} (o ninguna en caso de delivery).
  \item Un \textbf{Pedido} puede tener un registro \textbf{Delivery} asociado.
  \item Un \textbf{Pedido} puede tener múltiples registros de \textbf{Pago} (pago dividido).
  \item Un \textbf{Usuario} tiene un \textbf{Rol} que define sus permisos.
\end{itemize}

\section{Arquitectura del Sistema}

\subsection{Visión General de la Arquitectura}
El sistema se construye siguiendo los principios de Clean Architecture, que establece una separación clara de responsabilidades en diferentes capas:

\begin{itemize}
  \item \textbf{Capa de Presentación:} Componentes React para la interfaz de usuario y controladores de API en Django.
  \item \textbf{Capa de Aplicación:} Casos de uso e interactores que implementan la lógica de negocio específica.
  \item \textbf{Capa de Dominio:} Entidades del negocio y reglas empresariales fundamentales.
  \item \textbf{Capa de Infraestructura:} Implementaciones concretas de repositorios, acceso a bases de datos y servicios externos.
\end{itemize}

Esta arquitectura permite una alta cohesión y bajo acoplamiento, facilitando la mantenibilidad y evolución del sistema.

\subsection{Componentes y Módulos}
\begin{itemize}
  \item \textbf{Frontend:}
  \begin{itemize}
    \item Módulo de Autenticación y Control de Acceso
    \item Módulo de Gestión de Clientes
    \item Módulo de Gestión de Inventario
    \item Módulo de Gestión de Menús
    \item Módulo de Gestión de Pedidos
    \item Módulo de Gestión de Mesas
    \item Módulo de Delivery
    \item Módulo de Pagos
    \item Dashboard Analítico
  \end{itemize}
  
  \item \textbf{Backend:}
  \begin{itemize}
    \item API REST para todos los módulos
    \item Servicios de Autenticación y Autorización
    \item Servicios de Procesamiento de Pagos
    \item Servicios de Notificaciones
    \item Servicios de Generación de Reportes
    \item Servicios de Integración con Plataformas Externas
  \end{itemize}
  
  \item \textbf{Base de Datos:}
  \begin{itemize}
    \item PostgreSQL como motor principal
    \item Redis para caché y datos temporales
  \end{itemize}
\end{itemize}

\subsection{Integración con Sistemas Externos}
\begin{itemize}
  \item \textbf{Plataformas de Delivery:} Integración mediante APIs para recibir pedidos de plataformas externas.
  \item \textbf{Pasarelas de Pago:} Integración con servicios de procesamiento de pagos electrónicos.
  \item \textbf{Servicios de Mapas:} Para visualización y optimización de rutas de entrega.
  \item \textbf{Servicios de Correo:} Para envío de comprobantes y notificaciones.
  \item \textbf{Servicios SMS:} Para notificaciones a clientes sobre el estado de sus pedidos.
\end{itemize}

\subsection{Principios de Diseño y Patrones}
El sistema implementa diversos patrones de diseño para resolver problemas específicos:

\begin{itemize}
  \item \textbf{Repository Pattern:} Abstrae el acceso a datos, permitiendo cambiar la implementación sin afectar a los consumidores.
  \item \textbf{Factory Method:} Para crear diferentes tipos de objetos como pedidos locales vs. delivery.
  \item \textbf{Observer:} Para mantener sincronizados diferentes componentes cuando cambia el estado de entidades clave.
  \item \textbf{State:} Para gestionar los diferentes estados de pedidos y mesas.
  \item \textbf{Facade:} Para proporcionar una interfaz unificada para subsistemas complejos como el dashboard analítico.
  \item \textbf{Strategy:} Para implementar diferentes algoritmos de cálculo de precios o rutas de entrega.
  \item \textbf{Dependency Injection:} Para desacoplar componentes y facilitar las pruebas unitarias.
\end{itemize}

\section{Límites del Sistema}

\subsection{Alcance del Sistema}
El sistema abarca los siguientes módulos y funcionalidades:

\begin{itemize}
  \item Gestión completa de clientes, ingredientes, menús y pedidos.
  \item Administración de mesas y tiempos de servicio.
  \item Sistema de delivery con seguimiento de pedidos.
  \item Procesamiento de múltiples formas de pago.
  \item Dashboard analítico para toma de decisiones.
  \item Generación de comprobantes fiscales.
\end{itemize}

\textbf{No se incluye en esta fase:}
\begin{itemize}
  \item Gestión de proveedores y compras de inventario.
  \item Sistema de reservas online para clientes.
  \item Aplicación móvil dedicada para clientes finales.
  \item Integración con sistemas contables externos.
  \item Módulo de recursos humanos para gestión de personal.
\end{itemize}

\subsection{Integración con Servicios Externos}
\begin{itemize}
  \item El sistema se integrará con plataformas de delivery mediante APIs estándar.
  \item Se permitirá la integración con servicios de pago electrónico comunes.
  \item Se utilizarán servicios de mapas para visualización de rutas de entrega.
  \item La capacidad de integración con otros servicios estará limitada a interfaces documentadas y estables.
\end{itemize}

\subsection{Acceso de Usuarios}
\begin{itemize}
  \item El sistema soportará hasta 50 usuarios concurrentes en esta fase.
  \item El acceso se realizará mediante navegadores web modernos.
  \item Cada usuario accederá según su rol definido (Jefe de Local, Jefe de Turno, Mesero, Cocina).
  \item El acceso desde dispositivos móviles será mediante diseño responsivo, no aplicaciones nativas.
\end{itemize}

\subsection{Capacidad de Escalabilidad}
\begin{itemize}
  \item El sistema debe soportar un crecimiento anual del 30\% en volumen de datos.
  \item La arquitectura debe permitir escalar horizontalmente para manejar mayor carga.
  \item Las bases de datos deben optimizarse para manejar hasta 1000 transacciones por minuto.
  \item El diseño modular debe facilitar la adición de nuevas funcionalidades sin afectar las existentes.
\end{itemize}

\subsection{Compatibilidad de Dispositivos}
\begin{itemize}
  \item \textbf{Computadoras:} Compatibilidad con Windows, MacOS y Linux.
  \item \textbf{Navegadores:} Chrome, Firefox, Safari y Edge en sus versiones recientes.
  \item \textbf{Dispositivos móviles:} Tablets y smartphones con iOS 12+ y Android 8+.
  \item \textbf{Resoluciones:} Desde 1024x768 hasta 2560x1440 para PC, y estándares móviles.
\end{itemize}

\subsection{Soporte Multilenguaje}
\begin{itemize}
  \item La interfaz estará inicialmente en español.
  \item La arquitectura contemplará internacionalización (i18n) para futura expansión.
  \item Se utilizarán formatos estándar para fechas, moneda y notaciones numéricas.
\end{itemize}

\subsection{Respaldo y Recuperación}
\begin{itemize}
  \item Se realizarán backups automáticos diarios de la base de datos.
  \item Se mantendrán copias incrementales cada 6 horas.
  \item El tiempo máximo de recuperación (RTO) será de 1 hora.
  \item El punto objetivo de recuperación (RPO) será de máximo 6 horas.
\end{itemize}

\section{Planificación del Proyecto}

\subsection{Metodología}
El proyecto seguirá la metodología Scrum con 5 integrantes dedicando 5 horas semanales cada uno. Se estructurarán los siguientes sprints:
\begin{itemize}
  \item \textbf{Sprint 0:} 2 semanas para análisis y planificación inicial.
  \item \textbf{Sprint 1-3:} 4 semanas cada uno para desarrollo e implementación.
\end{itemize}

Cada sprint incluirá:
\begin{itemize}
  \item Sprint Planning al inicio
  \item Daily Standups (3 veces por semana)
  \item Sprint Review al final
  \item Sprint Retrospective para mejora continua
\end{itemize}

\subsection{Cronograma}
\begin{longtable}{|l|l|l|}
\hline
\textbf{Sprint} & \textbf{Objetivo Principal} & \textbf{Duración} \\
\hline
Sprint 0 & Análisis y diseño inicial & 2 semanas \\
\hline
Sprint 1 & Configuración de entornos y desarrollo inicial & 4 semanas \\
\hline
Sprint 2 & Desarrollo de funcionalidades principales & 4 semanas \\
\hline
Sprint 3 & Integración, pruebas y despliegue & 4 semanas \\
\hline
\end{longtable}

\subsection{Tareas del Sprint 0}

\subsubsection{1. Inventariar funcionalidades actuales}
\begin{itemize}
  \item Documentar operaciones CRUD y flujos relacionados con clientes, ingredientes, menús y pedidos.
  \item Identificar los tipos de informes y visualizaciones generados por el sistema actual.
\end{itemize}

\subsubsection{2. Mapear flujos de trabajo}
\begin{itemize}
  \item Crear diagramas detallados del proceso actual de toma de pedidos.
  \item Documentar el flujo actual de gestión de inventario.
\end{itemize}

\subsubsection{3. Identificar puntos débiles}
\begin{itemize}
  \item Analizar limitaciones de la interfaz de usuario actual.
  \item Identificar problemas de escalabilidad y rendimiento.
  \item Documentar las deficiencias en seguridad y gestión de usuarios.
\end{itemize}

\subsubsection{4. Definir arquitectura}
\begin{itemize}
  \item Establecer la estructura de capas siguiendo Clean Architecture.
  \item Seleccionar patrones de diseño para resolver problemas específicos.
  \item Definir interfaces y contratos entre componentes.
\end{itemize}

\subsubsection{5. Diseñar modelo de datos}
\begin{itemize}
  \item Crear el modelo entidad-relación completo.
  \item Definir las migraciones necesarias para crear la estructura.
  \item Planificar la estrategia de migración de datos del sistema actual.
\end{itemize}

\section{Resultados Esperados}

\subsection{Entregables Concretos}
\begin{itemize}
  \item \textbf{Aplicación web completa:} Frontend en React y backend en Django.
  \item \textbf{Base de datos:} Estructura completa y datos migrados del sistema actual.
  \item \textbf{Documentación:} Manuales técnicos y de usuario.
  \item \textbf{Material de capacitación:} Guías y videos tutoriales para usuarios.
  \item \textbf{Código fuente:} Repositorio Git con el código completo del proyecto.
  \item \textbf{Scripts:} Scripts de despliegue, migración y backup.
  \item \textbf{Pruebas automatizadas:} Suite de tests unitarios y de integración.
\end{itemize}

\subsection{Beneficios Cuantificables}
\begin{itemize}
  \item \textbf{Reducción del 90\%} en errores de pedidos.
  \item \textbf{Disminución del 40\%} en tiempo de servicio.
  \item \textbf{Aumento del 25\%} en la rotación de mesas.
  \item \textbf{Reducción del 30\%} en pérdidas de inventario.
  \item \textbf{Mejora del 35\%} en las calificaciones de satisfacción del cliente.
  \item \textbf{Incremento del 15\%} en margen de beneficio por mejor control de costos.
  \item \textbf{Reducción del 30\%} en horas-hombre dedicadas a tareas administrativas.
\end{itemize}

\subsection{Mejoras Operativas Esperadas}
\begin{itemize}
  \item \textbf{Mayor agilidad en atención al cliente:} Pedidos tomados directamente en mesa con dispositivos móviles.
  \item \textbf{Mejor comunicación entre áreas:} Notificaciones en tiempo real entre servicio y cocina.
  \item \textbf{Control efectivo de inventario:} Actualización automática de stock al procesar pedidos.
  \item \textbf{Optimización del espacio:} Mejor gestión de mesas y tiempos de ocupación.
  \item \textbf{Diversificación de canales de venta:} Integración con plataformas de delivery.
  \item \textbf{Toma de decisiones basada en datos:} Dashboard analítico con KPIs actualizados.
  \item \textbf{Mejora en la experiencia del cliente:} Reducción de tiempos de espera y errores.
\end{itemize}

\section{Apéndices}

\subsection{Glosario}
\begin{description}
  \item[Cliente:] Persona que realiza pedidos en el restaurante.
  \item[Pedido:] Conjunto de ítems solicitados por un cliente.
  \item[Menú:] Oferta del restaurante compuesta por uno o más ingredientes.
  \item[Ingrediente:] Unidad básica del inventario que compone los menús.
  \item[Mesa:] Espacio físico donde se atiende a los clientes en el restaurante.
  \item[Delivery:] Servicio de entrega de pedidos a domicilio.
  \item[Stock:] Cantidad disponible de un ingrediente en el inventario.
  \item[Nivel crítico:] Cantidad mínima de un ingrediente antes de generar alertas.
  \item[Dashboard:] Tablero visual con indicadores clave del negocio.
  \item[KPI:] Indicador clave de desempeño, métrica importante para el negocio.
  \item[Clean Architecture:] Patrón de arquitectura que separa responsabilidades en capas.
  \item[ORM:] Mapeo objeto-relacional, técnica para interactuar con la base de datos.
  \item[JWT:] JSON Web Token, mecanismo de autenticación segura.
  \item[API REST:] Interfaz de programación que sigue los principios RESTful.
  \item[Frontend:] Parte del sistema con la que interactúan directamente los usuarios.
  \item[Backend:] Parte del sistema que procesa las operaciones y gestiona los datos.
\end{description}

\subsection{Diagramas y Modelos}
\begin{itemize}
  \item Diagrama de Arquitectura (Ver documento anexo)
  \item Diagrama de Casos de Uso (Ver documento anexo)
  \item Diagrama Entidad-Relación (Ver documento anexo)
  \item Diagrama de Componentes (Ver documento anexo)
  \item Diagrama de Despliegue (Ver documento anexo)
\end{itemize}

\subsection{Casos de Uso}
Se documentan en detalle 33 casos de uso para el sistema refactorizado, organizados en módulos:
\begin{itemize}
  \item Gestión de Clientes: 5 casos de uso
  \item Gestión de Ingredientes: 5 casos de uso
  \item Gestión de Menús: 5 casos de uso
  \item Gestión de Pedidos: 5 casos de uso
  \item Gestión de Mesas: 3 casos de uso
  \item Sistema de Delivery: 3 casos de uso
  \item Sistema de Pagos: 3 casos de uso
  \item Dashboard Analítico: 3 casos de uso
\end{itemize}

Para cada caso de uso se detalla: actores, precondiciones, flujo principal, flujos alternativos, postcondiciones y excepciones.

Ver documento \texttt{Casos\_De\_Uso\_Extendidos.md} para la especificación completa.

\subsection{Anexos}
\begin{itemize}
  \item Mockups de interfaces principales
  \item Ejemplos de reportes analíticos
  \item Especificaciones técnicas detalladas
  \item Plan de pruebas y validación
  \item Plan de capacitación para usuarios
\end{itemize}

\end{document}
